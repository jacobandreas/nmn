\documentclass[10pt,twocolumn,letterpaper]{article}

% more motivation
% spatial heatmap / attention NOT detection
% not classify but recognize
% ``structured'' = easy
% how general is the parser? (where will it fail, are we losing perf., etc.)
% above vs not
% real figures for inplace

\usepackage{cvpr}
\usepackage{times}
\usepackage{epsfig}
\usepackage{graphicx}

\usepackage{booktabs}
%\usepackage{multicol}
\usepackage{caption}
\usepackage{subcaption}

%\usepackage{algorithm}
%\usepackage{algorithmic}
%\usepackage{hyperref}
%\usepackage{url}
%\newcommand{\theHalgorithm}{\arabic{algorithm}}
%
%\usepackage{latexsym}
%\usepackage{amsmath,amsfonts}
%\usepackage{amssymb}
%\usepackage{amsthm}
%\usepackage{relsize}
%\usepackage{mathtools}
%\usepackage{tikz}
%\usepackage{array}
%\usepackage{subcaption}
%\usepackage{comment}
%\usepackage{multirow}
%\usepackage{aliascnt}
%\usepackage{xspace}
%\usepackage[bb=fourier]{mathalfa}
%%\usepackage[font=small]{caption}
%\usepackage{siunitx}
%\usepackage{tablefootnote}
%\usepackage{multirow}
%\usepackage{microtype}
%\usepackage[export]{adjustbox}
%\usepackage{footmisc}

% Add a period to the end of an abbreviation unless there's one
 \makeatletter
 \DeclareRobustCommand\onedot{\futurelet\@let@token\@onedot}
 \def\@onedot{\ifx\@let@token.\else.\null\fi\xspace}
 \def\eg{e.g\onedot} \def\Eg{E.g\onedot}
 \def\ie{i.e\onedot} \def\Ie{I.e\onedot}
 \def\cf{cf\onedot} \def\Cf{Cf\onedot}
 \def\etc{etc\onedot} \def\vs{vs\onedot}
 \def\wrt{w.r.t\onedot} \def\dof{d.o.f\onedot}
 \def\etal{\textit{et~al\onedot}} \def\iid{i.i.d\onedot}
 \def\Fig{Fig\onedot} \def\Eqn{Eqn\onedot} \def\Sec{Sec\onedot}
 \def\vs{vs\onedot}
 \makeatother


\DeclareRobustCommand{\figref}[1]{Figure~\ref{#1}}
\DeclareRobustCommand{\figsref}[1]{Figures~\ref{#1}}

\DeclareRobustCommand{\Figref}[1]{Figure~\ref{#1}}
\DeclareRobustCommand{\Figsref}[1]{Figures~\ref{#1}}

\DeclareRobustCommand{\Secref}[1]{Section~\ref{#1}}
\DeclareRobustCommand{\secref}[1]{Section~\ref{#1}}

\DeclareRobustCommand{\Secsref}[1]{Sections~\ref{#1}}
\DeclareRobustCommand{\secsref}[1]{Sections~\ref{#1}}

\DeclareRobustCommand{\Tableref}[1]{Table~\ref{#1}}
\DeclareRobustCommand{\tableref}[1]{Table~\ref{#1}}

\DeclareRobustCommand{\Tablesref}[1]{Tables~\ref{#1}}
\DeclareRobustCommand{\tablesref}[1]{Tables~\ref{#1}}

\DeclareRobustCommand{\eqnref}[1]{Equation~(\ref{#1})}
\DeclareRobustCommand{\Eqnref}[1]{Equation~(\ref{#1})}

\DeclareRobustCommand{\eqnsref}[1]{Equations~(\ref{#1})}
\DeclareRobustCommand{\Eqnsref}[1]{Equations~(\ref{#1})}

\DeclareRobustCommand{\chapref}[1]{Chapter~\ref{#1}}
\DeclareRobustCommand{\Chapref}[1]{Chapter~\ref{#1}}

\DeclareRobustCommand{\chapsref}[1]{Chapters~\ref{#1}}
\DeclareRobustCommand{\Chapsref}[1]{Chapters~\ref{#1}}


% \setlength{\topskip}{0mm}
% \setlength{\abovecaptionskip}{3mm}
% \setlength{\belowcaptionskip}{3mm}
% \setlength{\textfloatsep}{5mm}

\hyphenation{po-si-tive}
% Include other packages here, before hyperref.

%\captionsetup{belowskip=8pt}

% If you comment hyperref and then uncomment it, you should delete
% egpaper.aux before re-running latex.  (Or just hit 'q' on the first latex
% run, let it finish, and you should be clear).
\usepackage[pagebackref=true,breaklinks=true,letterpaper=true,colorlinks,bookmarks=false]{hyperref}

% \cvprfinalcopy % *** Uncomment this line for the final submission

\def\cvprPaperID{1114} % *** Enter the CVPR Paper ID here
\def\httilde{\mbox{\tt\raisebox{-.5ex}{\symbol{126}}}}

% Pages are numbered in submission mode, and unnumbered in camera-ready
\ifcvprfinal\pagestyle{empty}\fi
\begin{document}

\newcommand{\nmn}{Neural Module Networks\xspace}
\newcommand{\mod}[1]{{\small\texttt{#1}}}
\newcommand{\todo}{\textcolor{red}}
\newcommand{\shapes}{{\textsc{shapes}}\xspace}
\newcommand{\cocoqa}{{\textsc{CocoQA}}\xspace}
\newcommand{\marcus}[1]{\textcolor{blue}{Marcus: #1}}
%%%%%%%%% TITLE
\title{Deep Compositional Question Answering with Neural Module Networks}

\author{First Author\\
Institution1\\
Institution1 address\\
{\tt\small firstauthor@i1.org}
% For a paper whose authors are all at the same institution,
% omit the following lines up until the closing ``}''.
% Additional authors and addresses can be added with ``\and'',
% just like the second author.
% To save space, use either the email address or home page, not both
\and
Second Author\\
Institution2\\
First line of institution2 address\\
{\tt\small secondauthor@i2.org}
}

\maketitle
%\thispagestyle{empty}

%%%%%%%%% ABSTRACT
\begin{abstract}
  Visual question answering differs from many other vision tasks in that the
  computational structure necessary to produce an answer is not fixed, but
  instead varies depending on the question. \marcus{I am confused with this sentence. It is stated like known fact, but prior work says differently\ldots and that what we want to show in the paper.} 
  At the same time, the task is
  fundamentally compositional in nature---answering a question like \emph{where
  is the dog?} shares substructure with questions like \emph{what color is the
  dog?} and \emph{where is the cat?}. This paper seeks to simultaneously exploit the
  representational capacity of deep networks and the dynamic compositional
  structure of questions.  We describe a procedure for constructing and learning
  \emph{neural module networks}, which compose collections of jointly-trained
  neural ``modules'' into deep networks for question answering. Our approach
  decomposes questions into their linguistic substructures, and uses these
  structures to dynamically instantiate modular networks (with reusable
  components for recognizing dogs, classifying colors, etc.). The resulting
  compound networks are jointly trained. We evaluate our approach on three
  challenging datasets for visual question answering, achieving several
  state-of-the-art results.
%Answering a question about an image, is a challenging task which requires understanding the question, grounding it in the image and jointly reasoning about the image content to produce a correct answer. Intuitively, this task is compositional in nature, \ie answering \emph{where the dog} in an image is can partially share an representation with \emph{which color the dog} is or \emph{where the cat} is. On the other hand, learning deep neural networks has shown impressive progress in question answering and related tasks. Consequently, we propose \emph{\nmn}, which composes several, jointly trained neural modules to answer questions about images. Specifically our approaches decomposes questions into their linguistic substructures and instantiates corresponding neural ``modules'' (\eg \emph{where, which color, dog}) which are stacked to a deep network. The resulting compound
%  networks are then jointly trained, whose composable neural ``modules'' 
%  can be dynamically remixed to interpret also novel questions. We evaluate our approach
%  on three challenging datasets for visual question answering, achieving
%   overall state-of-the-art performance, but outperforming related work for more complex questions.
\end{abstract}

\section{Introduction} 

\begin{figure}[t] \begin{center}
    \includegraphics[width=\linewidth]{fig/teaser} \end{center} \caption{Our
    approach answers questions about images, it automatically instantiates
    different modules of neural networks depending on the question. An
    additional LSTM provides sentence context and learns common sense / dataset
  bias.} \label{fig:teaser}
\end{figure}

This paper describes a framework for answering natural language questions with
collections of jointly-trained neural ``modules'', using linguistic structure to
dynamically assemble these modules into deep networks. Our initial focus is on
visual question answering, a task with significant applications to human-robot
interaction, search, and accessibility. Visual QA is the subject of a great deal
of current research attention 
\cite{antol15iccv,gao2015you,ma15arxiv,malinowski15iccv,ren2015image,yu15arxiv}, 
requiring sophisticated understanding of both visual scenes and natural
language.

Specifically, given an image and an associated question (e.g.\ \emph{where is
the dog?}), we wish to predict a corresponding answer (e.g.\ \emph{on the couch},
or perhaps just \emph{couch}). Recent successful represent questions as
bag-of-words \cite{} or read in the question using a single neural network
\cite{malinowski15iccv}\cite{} and train a classifier on the full question
representation and the full-image representation. In contrast to these
monolithic approaches, another line of work for textual QA \cite{Liang13DCS} and
image QA \cite{malinowski14nips} uses semantic parsers to decompose questions
into logical expressions. These logical expressions are evaluated against a
logical representation of the world, which may be provided directly or extracted
from an image \cite{Krish2013Grounded}.

In this paper we draw from both lines of research, presenting a technique for
integrating the representational power of neural networks with the flexible
compositional structure afforded by symbolic approaches to semantics.  Rather
than relying on a monolithic network structure to answer all questions, our
approach assembles a network on the fly from a collection of specialized,
jointly-learned modules (\autoref{fig:teaser}). In particular, we first analyze
the question with an off-the-shelf semantic parser, and use this analysis to
determine the basic computational units (detection, classification, etc.) needed
to answer the question, as well as the relationships between them. In
\autoref{fig:teaser}, we first produce a dog detector, which sends its output to
a location classifier. Depending on the underlying structure, these messages
passed between modules may be raw image features, attentions, or classification
decisions; each module is determined by its input and output types.
Different kinds of modules are marked with different
colors in \Figref{fig:teaser}: for example the \mod{attend[$dog$]} module
(green) produces a spatial heatmap while the \mod{classify[$where$]} (blue)
produces a classification output, given an attention heatmap and the image
content. Importantly, all modules of a NMN are independent, which allows the
computation to be different for each problem instance, and possibly unobserved
during training. 
%So that we can answer novel question at test time, such as
%\emph{Where is the banana?}, even we only saw \emph{count} or
%\emph{color} question about \emph{bananas} during training.  
In addition to the NMN our final answer also incorporates the image scene
knowledge (using a full frame CNN) and uses an  recurrent network (LSTM) to read
the question, which has been shown to be important to model common sense
knowledge and dataset biases \cite{malinowski15iccv}.

%Where previous work has
%treated both the image and the question as inputs to a monolithic
%classification model, we instead take the perspective that a question is a
%noisy specification of a hidden computation that must be performed on the image
%to produce an answer. Crucially, this computation may be different for each
%problem instance, and is never observed observed during training.

%Our approach bears a superficial resemblance to a classical semantic parser.
%However, instead of mapping from questions to logical forms, our model maps
%from questions to neural network structures. These networks are assembled on
%the fly (possibly into novel topologies) from a collection of jointly-learned
%neural ``modules''. Finally, they are evaluated against the input image to
%produce an answer.




%This paper presents a technique for following natural language instructions
%(and performing other dynamically-specified tasks) by assembling deep neural
%networks on the fly from an inventory of pre-trained components.

We evaluate our approach on three visual question answering tasks. On the
recently-released CocoQA \cite{yu15arxiv} and  VQA \cite{antol15iccv} datasets,
we achieve results comparable to [better than] existing approaches, and show
that our approach specifically outperforms previous work on questions with
compositional structure (\eg requiring that an object be located and one of its
attributes described). Using our model on questions where compositional
structure plays a central role, and falling back to previous approaches
elsewhere, we achieve new state-of-the-art results on both tasks. It turns out,
however, that most of the questions in both datasets are quite simple, with
little composition. To test our approach's ability to handle highly structured
questions, we introduce a new dataset of synthetic images paired with complex
questions involving spatial relations, set-theoretic reasoning, and shape and
attribute recognition. On this dataset we outperform the previous state of the
art by XXX.

While all the applications considered in this paper involve visual question
answering, the general architecture is potentially of broader usefulness, and
might be more generally applied to referring expression resolution (XXX),
question answering about natural language texts (XXX), or XXX.

%We evaluate our approach on two visual question answering tasks. First we
%present a new synthetic image dataset paired with a complex set of queries
%(involving spatial relations, logical operators, and shape and attribute
%recognition). Next, we consider a hard subset of the Microsoft VQA corpus of
%questions about natural images. In each case, an NMN-based approach outperforms
%state-of-the-art models with more conventional recurrent architectures. We
%observe in particular that NMNs are able to make considerably better use of
%small training sets.

To summarize our contributions: We first propose neural module networks, a
general architecture for discretely composing heterogeneous, jointly-trained
neural modules into deep networks. Next, for the visual QA task specifically, we
show how to construct NMNs based on the output of a semantic parser, and use
these to successfully complete established visual question answering tasks.
Finally, we introduce a new dataset of challenging, highly compositional
questions about abstract shapes, and show that our approach outperforms previous
models by as much as XXX. We will release this dataset, as well as all code for
our project, at $<anonymous>$.


\section{Motivations}

Many tasks in computer vision, including recognition, detection, and captioning,
share common substructure. For example, we might schematically express the
sequence of computations performed by a recognizer as
\begin{flushleft}
  \mod{classify(pickMostRelevant(detectObjects))}
\end{flushleft}
or a detector as
\begin{flushleft}
  \mod{drawBoundaries(detectObjects)}
\end{flushleft}

In practice the picture is not this clean---classification or detection is
performed end-to-end by a single neural network, and the boundaries between
these ``phases'' are not clearly defined. Nevertheless we might expect \textit{a
priori} that a network used for classification might expose intermediate
representations useful for building a detector. Indeed, it is now commonplace to
initialize systems for a variety of vision tasks with a prefix of a network
trained for classification \cite{Long14FullyConvolutional}. This has been shown
to substantially reduce training time and improve accuracy. So while network
structures are not \emph{universal} (in the sense that the same network is
appropriate for all problems), they are at least empirically \emph{modular} (in
the sense that intermediate representations for one task are useful for many
others). 

Can we generalize this idea in a way that is useful for question answering?
Rather than thinking of question answering as a problem of learning a single
function to map from questions and contexts to answers, it's perhaps useful to
think of it as a highly-multitask learning setting, where each problem instance
is associated with a novel task, and the identity of that task is expressed only
noisily in language. If we consider a few examples of questions:
XXX
%\begin{center}
%  \begin{tabular}{ll}
%    {\it how many black cats?} & \mod{count(and(detect[cat], detect[black]))} \\
%    {\it what color is the cat?} & \mod{classify[color](detect[cat])} \\
%    {\it what color is the dog?} & \mod{classify[color](detect[dog])} \\
%    {\it is there a dog?} & \mod{exists(detect[dog])}
%  \end{tabular}
%\end{center}
we again see that there is common computational substructure involved in solving
the associated tasks.  With sub-networks for computing \mod{detect[cat]},
\mod{classify}, \mod{count}, etc., we can in principle answer
questions with novel structure like---e.g.\ {\it is there a black
dog?}---without any additional training data.

Note in particular that we expect these modules to differ not only in their
parameters, but more fundamentally in their topologies. Intuitively,
\mod{detect[cat]} should take an image as input, perform some
fully-convolutional operation, and output an attention (understood as a
distribution over positions in the image), while {\small\tt classify[color]}
should take both the input image and such an attention, and map to a
distribution over labels. Some computations require convolutional operations,
some require fully-connected operations, and some (like counting) may require
recurrent network structures. We should not expect that we will be able to use
the same network layout for every problem, but do expect that parts of these
networks may be reused in different orders.

Thus our goal in this paper is to specify a framework for modular, composable,
jointly-trained neural networks. In this framework, we first predict the
structure of the computation needed to answer each question individually, then
realize this structure by constructing an appropriately-shaped neural network
from an inventory of reusable modules. These modules are learned jointly, rather
than trained in isolation, and specialization to individual tasks (identifying
properties, spatial relations, etc.) arises naturally from the training
objective.

\section{Related work}
We discuss related work which are similar with respect to the task we like to solve, namely visual question answering, and then with respect to the network architecture, i.e.\ similarities to our proposed Neural Module Networks.

\subsection{Visual Question Answering}
Answering questions about images also has been referred to as Visual Turing Test \cite{malinowski14nips,geman15nas}. It is a task which only recently started to become popular with the emergence of visual question answering datasets, which have image, question and answer triplets. %The present work is made possible by a number of recently-created corpora for question answering about images, including the 
While the DAQUAR dataset \cite{malinowski14nips} is restricted to indoor scenes, the CocoQA dataset \cite{yu15arxiv} and the VQA dataset \cite{antol15iccv} are significantly larger and have more visual variety. Both are based on images from the COCO dataset \cite{lin14eccv}. While CocoQA contains question-answer pairs automatically generated from the descriptions associated with the COCO dataset, \cite{antol15iccv} has crowed sourced questions-answer pairs. We evaluated our approach on the latter two.



``classical'': \cite{malinowski14nips}. Similar: Parser to model, but computation not composed, monolithic, not end-to-end trainable.

``neural'': not decompositional.
Several models for visual questioning have already been proposed in the literature \cite{ren2015image,ma15arxiv,gao2015you}, all of which use standard deep sequence modeling machinery to construct a joint embedding of image and text, which is immediately mapped to a distribution over answers. Here we attempt to more XXXly model the computational process needed to produce each answer, but also benefit from sequence and image embeddings XXX.

A part of solving visual question answering, is to ground the question in the image. This grounding task has previously been approached in \cite{karpathy14nips,plummer15iccv,karpathy15cvpr}, where the authors tried to localize phrases in an image. \cite{xu2015arxiv} use an attention mechanism, to predict a heatmap for each word, as an auxiliary task, during sentence generation. Our attention modules are inspired by this idea.
%(Image Retrieval using Scene Graphs \cite{johnson15cvpr})
%\cite{geman15nas} a binary (yes/no) version of Visual Turing Test on synthetic data.

Answering natural question has previously also been studied in text-only or non-visual/synthetic data.
\todo{describe\cite{berant14acl,Liang13DCS,iyyer14emnlp,weston14arxiv}} 

We are also unaware of past use of a semantic parser to predict network structures, or more generally to exploit the natural similarity between set-theoretic approaches to classical semantic parsing and attentional approaches to computer vision. As noted above, there is a large literature on learning to answer questions about structured knowledge representations from question--answer pairs, both with and without learning of meanings for simple predicates \cite{Liang13DCS,Krish2013Grounded}.


\subsection{Neural network architectures}

The idea of selecting a different network graph for each input datum is fundamental to both recurrent networks (where the network grows in the length of the input) \cite{Elman90RNN} and recursive neural networks (where the network is built, e.g., according to the syntactic structure of the input) \cite{Socher13CVG}. But both of these approaches ultimately involve repeated application of a single computational module (\eg an LSTM \cite{} or GRU \cite{} cell). Our basic contribution is in allowing the nodes of this graph to perform heterogeneous computations, and for "messages" of different kinds---raw image features, attentions, classification predictions---to be passed from one module to the next. We are unaware of any previous work allowing such mixed collections of modules to be trained jointly. 

Memory networks \cite{weston14arxiv}?

\paragraph{TODO}
XXX my conll and emnlp for "continuous on bottom, discrete on top" with Jayant

\cite{Krish2013Grounded,matuszek12icml,kong14cvpr}



\section{Model}

Given a pre-trained \emph{network layout predictor} $P$, each training datum for
this task can be thought of as a 4-tuple $(w, p, x, y)$, where
\begin{itemize}
  \setlength\itemsep{0em}
  \item $w$ is a natural-language question
  \item $p = P(w)$ a network layout
  \item $x$ is an image
  \item $y$ is an answer
\end{itemize}
A model is fully specified by a collection of modules $\{ m \}$, each with
associated parameters $\theta_m$. Given $(w, p, x)$ as above, the model
instantiates a network according to $p$, passes $w$ and $x$ as inputs, and
obtains a distribution over labels (for the VQA task, we require the root module
to be a classifier). Thus a model corresponds to a predictive distribution $p(y\
|\ w, p, x; \theta)$.

In the remainder of this section, we first describe the set of modules used for
the VQA task, then explain the process by which questions are converted to
network layouts.

\subsection{Modules}

Our goal in this section is to identify a small set of modules that can be
assembled into all the configurations necessary for our tasks. This corresponds
to identifying a minimal set of composable vision primitives. The modules
operate on three basic data types: images, unnormalized attentions, and labels.
For the particular task and modules described in this paper, almost all
interesting compositional phenomena occur in the space of attentions, and it is
not unreasonable to characterize our contribution more narrowly as an
``attention-composition'' network. Nevertheless, other types may be
required in the future (for different new or for greater coverage in the VQA
domain).

First, some notation: module names are typeset in a {\tt fixed width font}, and
are of the form \mod{TYPE[INSTANCE](ARG$_1$, \ldots)}. \mod{TYPE} is a
high-level module type (attention, classification, etc.) of the kind described
in this section. \mod{INSTANCE} is the particular instance of the model under
consideration---for example, \mod{attend[red]} locates red things, while
\mod{attend[dog]} locates dogs. Weights may be shared at both the type and
instance level. Modules with no arguments implicitly take the image as input;
higher-level arguments may also inspect the image.

\paragraph{Attend}

An attention module \mod{attend[$x$]} convolves every position in the input
image with a weight vector (distinct for each $x$) to produce a heatmap or
unnormalized attention. So, for example, the output of the module {\small\tt
attend[dog]} is a matrix whose entries should be in regions of the image
containing cats, and small everywhere else, as shown below.\\
\includegraphics[width=\columnwidth]{fig/attend}

\paragraph{Re-attend}

A re-attention module \mod{re-attend[$x$]} performs a fully-connected mapping
from one attention to another. Again, the weights for this mapping are distinct
for each $x$. So \mod{re-attend[above]} should take an attention and shift the
regions of greatest activation upward (as below), while \mod{re-attend[not]}
should move attention away from the active regions.\\%[1em]
\includegraphics[width=\columnwidth]{fig/re-attend}

\paragraph{Combine}

A combination module \mod{combine[$x$]} merges two attentions into a single
attention. For example, {\small\tt combine[and]} should be active only in the
regions that are active in both inputs, while {\small\tt{combine[except]}}
should be active where the first input is active and the second is
inactive.\\[1em] 
\includegraphics[width=\columnwidth]{fig/combine}

\paragraph{Classify}

A classification module \mod{classify[$x$]} takes an attention and the input
image and maps them to a distribution over labels. For example, {\small\tt
classify[color]} should return a distribution over colors for the region
attended to.\\[1em]
\includegraphics[width=\columnwidth]{fig/classify}

\subsection{From strings to networks}

In this section, we first describe how to map from natural language questions to
\emph{layouts}, which specify both the set of modules used to answer a given
question, and the connections between them. Next we describe how layouts are
used to assemble the final prediction networks.

To predict layouts use standard tools pre-trained on existing linguistic
resources to obtained structured representations of questions. Future work might
focus on learning (or at least fine-tuning) this prediction process jointly with
the rest of the system, though in fact off-the-shelf tools produce high-quality
analyses for almost all the questions in our data.

\paragraph{Parsing}
Specifically, we begin by parsing each question with the Stanford Parser (XXX)
to obtain a universal dependency representation (XXX). Dependency parses express
grammatical relations between parts of a sentence (e.g.\ between objects and
their attributes, or events and their participants), and provide a lightweight
abstraction away from the surface form of the sentence.

Next, we filter the set of dependencies to those involving the wh-word in the
question (XXX and more). This gives a simple logical form expressing (the
primary) part of the sentences meaning. For example, \emph{what is standing in
the field} becomes \mod{what(stand)}; \emph{what color is the truck} becomes
\mod{color(truck)}, and \emph{is there a circle next to a square} becomes
\mod{is(cicle, next-to(square))}. It is easiest to think of these
representations as pieces of a variable-free cominatory logic \cite{Liang13DCS};
every leaf is implicitly a function taking the image as input. The code for
transforming parse trees to structured queries is provided in the accompanying
software package.

\paragraph{Layout}
These logical forms already determine the structure of the predicted network,
but not the identities of the modules that compose it. This final assignment of
modules is fully determined by the structure of the parse. All leaves become
\mod{attend} modules, all internal nodes become \mod{re-attend} or \mod{combine}
modules dependint on their arity, and root nodes become \mod{classify} modules.

Given the mapping from queries to network layouts described above, we have for
each training example a network structure, an input image, and an output label.
In many cases, these network structures are different, but have tied parameters.
Networks which have the same high-level structure but different instantiations
of individual modules (for example \emph{what color is the cat}---{\small\tt
classify[color](attend[cat])} and \emph{where is the truck}---{\small\tt
classify[where](attend[truck])}) can be processed in the same batch, resulting
in efficient computation.

\paragraph{Generalizations}

It is easy to imagine applications where the input to the layout stage comes
from something other than a natural language parser. Users of an image database,
for example, might write SQL-like queries directly in order to specify their
requirements precisely ({\tt EXISTS(cat) AND NOT(EXISTS(dog))}), or even mix
visual and non-visual specifications in their queries ({\tt EXISTS(cat) and DATE
> 2014-11-5}).

Indeed, it is possible to construct this kind of ``visual SQL'' using exactly
the models learned in this paper---once our system is trained, the learned
modules for attention, classification, etc. can be assembled by any kind of
outside user, without relying on natural language specifically.

\begin{figure*}
  \begin{subfigure}[t]{0.4\textwidth}
    \includegraphics[width=\textwidth]{fig/full1}
    \caption{NMN for answering the question \emph{What color is
    his tie?}. The \mod{attend[tie]} module first predicts a heatmap
    corresponding to the location of the tie. Next, the \mod{classify[color]}
    module uses this heatmap to produce a weighted average of image features,
  which are finally used to predict an output label.}
  \end{subfigure}
  \hfill
  \begin{subfigure}[t]{0.55\textwidth}
    \includegraphics[width=\textwidth]{fig/full2}
    \caption{NMN for answering the question \emph{Is there a red shape above a
    circle?}. The two \mod{attend} modules locate the red shapes and circles,
    the \mod{re-attend[above]} shifts the attention above the circles, the
    \emph{combine} module computes their intersection, and the
    \mod{exists} module inspects the final attention and determines that it is
    non-empty.}
  \end{subfigure}
  \caption{Sample NMNs for question answering about natural images and shapes.}
\end{figure*}

\section{Learning}

Our training objective is simply to find module parameters maximizing the
likelihood of the data. By design, the last module in every network is a
{\small\tt classify}, and so each assembled network represents a probability
distribution.

Because of the dynamic network structures used to answer questions, some weights
are updated much more frequently than others. For this reason we found that
learning algorithms with adaptive per-weight learning rates performed
substantially better than simple gradient descent. All the experiments described
below use AdaDelta (XXX) (thus there was no hyperparameter search over step
sizes).

It is important to emphasize that the labels we have assigned to distinguish
instances of the same module type---\mod{cat}, \mod{and}, etc.---are a
notational convenience, and do not reflect any manual specification of the
behavior of the corresponding modules. \mod{detect[cat]} doesn't know it's
supposed to be a cat recognizer (rather than a couch recognizer or a dog
recognizer),
and \mod{combine[and]} doesn't know it's supposed to compute intersections of
attentions (rather than unions or differences). Instead, they acquire these
semantics as a byproduct of the end-to-end training procedure. As can be seen in
Figure XXX, the image--answer pairs and parameter tying together encourage each
module to specialize in the appropriate way.

%---a simple feedforward
%convolutional network is suitable for most detection and classification tasks,
%but counting to arbitrary numbers probably requires a recurrent network.


\section{Experiments: natural images}

For both experiments in this section, the input to each \mod{attend} module is
the the Conv XXX layer of an Oxford VGGNet (XXX) after max-pooling. We do not
fine-tune the VGGNet.

\paragraph{VQA}
We begin with a set of experiments on the recently-released VQA dataset. This
dataset consists of more than 200,000 images, each paired with three questions
and ten answers per question. Data was generated by XXX, more corpus flavor
text. We train our model using the standard train/test split, training only with
those answers marked as high confidence.

Results are shown in Table XXX. As can be seen, we achieve state-of-the-art
results on this task. A breakdown of our questions by answer type reveals that
our model performs especially well on questions answered by an object or
attribute, but worse than a sequence baseline on the other two categories.
Inspection of training-set accuracies suggests that performance in these
categories is due to overfitting. A hybrid system which falls back to an NMN
for these ``other'' category achieves even better results; future work within
the NMN framework might focus on redesigning the \mod{classify-attention} module
so it is less prone to overfitting.

\begin{table}
  \footnotesize
  \center
  \begin{tabular}{lccccc}
    \toprule
    & \multicolumn{4}{c}{test-dev} & test \\
    \cmidrule(lr){2-5} \cmidrule(lr){6-6}
    & All & Yes/No & Number & Other & All \\
    \midrule
    LSTM & 48.8 & 78.20 & 35.7 & 26.6 \\
    VIS+LSTM & 53.7 & 78.9 & 35.2 & 36.4 & \\
    NMN & \\
    \bottomrule
  \end{tabular}
  \caption{VQA results}
\end{table}

\paragraph{\cocoqa}
We also evaluate our system on the smaller \cocoqa dataset introduced by Ren
XXX. This is similar in spirit to the VQA corpus, but uses questions
automatically generated from captions rather than directly elicited from users.
The \cocoqa dataset features four categories of question: objects, locations,
colors and numbers. Our results are shown in Table XXX. As can be seen, the NMN
model substantially outperforms the baselines in the color category, but is
slightly worse in the others (this again appears to be due to overfitting). As
above, if we replace our predictions in the count category only with a simple
text baseline, we again achieve new state-of-the-art results on this task.

\begin{table}
  \footnotesize
  \center
  \begin{tabular}{cccccc}
    \toprule
    System & All & Object & Location & Color & Number \\
    \midrule
    IMG+BOW & 55.9 & 58.7 & 49.4 & 52.0 & 44.1 \\
    2-VIS+BLSTM & 55.1 & 58.2 & 47.3 & 49.5 & 44.8 \\
    NMN (indep) \\
    NMN & 55.5 & 57.9 & 48.8 & 54.3 & 40.5 \\
    \bottomrule
  \end{tabular}
  \caption{\cocoqa results}
\end{table}

\section{Experiments: compositionality}

Past work [Ren paper] has achieved state-of-the-art results on the \cocoqa
dataset using a bag-of-words model . This is consistent with [VQA paper]'s
observation that image features are most important for simple object- and
activity-recognition questions, and that they make little difference for other
categories (like yes/no questions). Taken together, these results suggest that
existing natural image datasets primarily require simple object- and attribute
recognition, with limited importance attached to spatial relations and other
highly compositional phenomena.

As one of the primary goals of this work is to learn models for deep semantic
compositionality, we have created an additional dataset that places such
compositional phenomena at the forefront. This dataset consists of complex
questions about simple arrangements of colored shapes [XXX].  Questions contain
between two and four attributes, object types, or relationships.  There are 244
questions and 15616 images in total, divided into train and test sets (with no
overlap between the train and test sets).  To eliminate mode-guessing as a
viable strategy, all questions have a yes-or-no answer, but good performance
requires that the system learn to recognize shapes and colors, and understand
both spatial and logical relations among sets of objects.

XXX babyai

\begin{figure}
  \centering
  \emph{Is there a red shape above a circle?} \\[1em]
  \begin{tabular}{ccc}
    \includegraphics[width=0.25\columnwidth]{fig/shapes1_big} &
    \includegraphics[width=0.25\columnwidth]{fig/shapes2_big} &
    \includegraphics[width=0.25\columnwidth]{fig/shapes3_big} \\
    no & yes & yes
  \end{tabular}
  \caption{Example question, images, and answers from the \shapes dataset.
    Answering this question requires being able to recognize an object
    (\emph{circle}), an attribute (\emph{red}), and a spatial relation
    (\emph{above}), and compare sets of objects having all these properties.}
\end{figure}


%\begin{figure}
%  \centering
%  \begin{tabular}{cc}
%    %\includegraphics[width=0.25\columnwidth]{fig/shapes_in1} &
%    %\includegraphics[width=0.25\columnwidth]{fig/shapes_out1} \\
%    \includegraphics[width=0.25\columnwidth]{fig/shapes_in2} &
%    \includegraphics[width=0.25\columnwidth]{fig/shapes_out2} \\
%  \end{tabular}
%\end{figure}

\begin{table}
  \footnotesize
  \centering
  \begin{tabular}{ccccc}
    \toprule
    System & All & Size 2 & Size 3 & Size 4 \\
    \midrule
    IMG+BOW & \\
    VIS+LSTM &  \\
    NMN & 90.62* & 89.69* & 92.36* & 85.16* \\
    \bottomrule
  \end{tabular}
  \caption{Synth data results}
\end{table}

As can be seen, our model achieves excellent performance on this dataset, while
competing approaches fare little better than the majority baseline. Moreover,
the color detectors and attention transformations behave as expected, indicating
that our joint training procedure correctly allocates responsibilities among
modules. Ultimately, in addition to achieving competitive results in answering
simple questions about natural images, our approach is able to model complex
compositional phenomena outside the capacity of previous approaches to visual
question answering.

\section{Conclusions and future work}

In this paper, we have introduced \emph{neural module networks}, which provide a
general-purpose framework for learning collections of neural modules which can
be dynamically assembled into arbitrary deep networks. We have demonstrated that
this approach achieves state-of-the-art performance on existing datasets for
visual question answering. Additionally, we have introduced a new dataset of
highly compositional questions about simple arrangements of shapes, and shown
that our approach substantially outperforms previous work.

So far we have maintained a strict separation between predicting network
structures and learning network parameters. It is easy to imagine that these two
problems might be solved jointly, with uncertainty maintained over network
structures throughout training and decoding. This might be accomplished either
XXX.

The fact that deep neural modules can be trained to produce predictable
outputs---even when freely composed---points toward a more general paradigm of
``programs'' built from neural networks. In this paradigm, network designers
(human or automated) have access to a standard kit of neural parts from which to
construct models for performing complex reasoning tasks. While visual question
answering provides a natural testbed for this approach, its usefulness is
potentially much broader, extending to queries about documents and structured
knowledge bases or more general signal processing and function approximation.

\small
\bibliographystyle{ieee}
\bibliography{biblioShort,rohrbach,related,jacob}


\end{document}
